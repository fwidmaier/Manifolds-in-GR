\documentclass[a4paper, 12pt]{article}

\usepackage[english]{babel}
\usepackage{hyperref}
\usepackage{url}
\usepackage[T1]{fontenc}
\usepackage{mlmodern}
\usepackage{tikz}
\usepackage{float}

\usepackage{sectsty}
\allsectionsfont{\sffamily\mdseries\upshape\bfseries}

\usepackage[margin=2.5cm]{geometry}
\usepackage{amsmath}
\usepackage{amsfonts}
\usepackage{amssymb}
\usepackage{amsthm}
\usepackage{graphicx}
\usepackage{enumerate}

\usepackage{biblatex}

\bibliography{bibliography} 

\newtheorem{definition}{Definition}
\newtheorem{theorem}{Theorem}
\newtheorem{lemma}{Lemma}[section]
\newtheorem{corollary}{Corollary}[theorem]

\theoremstyle{iremark}

\makeatletter
\renewcommand\@endtheorem{\vvv@endmarker\endtrivlist\@endpefalse}
\newcommand\vvv@endmarker{%
  {\unskip\nobreak\hfil\penalty50
  \hskip2em\vadjust{}\nobreak\hfill$\blacktriangleleft$
  \parfillskip=0pt \finalhyphendemerits=0 \par
  \penalty 10000 \parskip=0pt\noindent}\ignorespaces}
\makeatother


\newtheorem{claim}{Claim}

\newcommand{\R}{\mathbb{R}}
\newcommand{\Z}{\mathbb{Z}}
\renewcommand{\d}{\mathrm{d}}

\title{\textbf{\textsf{Lost in space and time: Why topological considerations matter for spacetime}}}
\author{Felix Widmaier\thanks{\href{mailto:felixwidmaier [at] gmx.de}{\texttt{felixwidmaier[at]gmx.de}}}}
\date{December 2024}

\begin{document}
\maketitle
\begin{abstract}
    \noindent In the lecture ‘Introduction to General Relativity’ by Jens Niemeyer in the winter semester 2024/25, manifolds 
    have been introduced as spaces which are locally Euclidean and which have a smooth differentiable structure. 
    This definition differs significantly from the ‘classical’ definition in the mathematical literature. 
    There, it is often additionally required that manifolds are paracompact and Hausdorff. \cite{jost} After the lecture, 
    I conjectured that these topological constraints could be implicitly inferred. However, this assumption is wrong. 
    There are certainly topological spaces which are neither Hausdorff nor paracompact, but which fulfil the above 
    ‘weaker’ definition of a manifold. This results in considerable restrictions for the geometry and analysis on such 
    manifolds (and thus also for physics).
\end{abstract}

\section{Topological pathologies and pathological topology}
Topology, like differential geometry, is a very rich and beautiful field of mathematics that is often riddled with 
counterintuitive results\footnote{see for example \cite{counter} for a whole book about counter examples in topology} 
that make topology nevertheless enjoyable and hence even worthier of thorough study. For example, the notion of a manifold 
has gradually become more sophisticated in mathematics and makes use of some topological constraints on the underlying 
topological spaces, which may not seem particularly intuitive at first.
A rather classical definition of manifolds from \cite{jost} is
\begin{definition}[Manifold, \cite{jost}]
    A manifold is a connected paracompact Hausdorff and locally Euclidean space.
\end{definition}
These topological restrictions are imposed for good reasons, to exclude some pathological spaces that are not suitable for 
modelling (or simply for aesthetic reasons). Nevertheless, it may be a fruitful endeavour to explore why these constraints 
are imposed on the underlying spaces - and to what extent the notion of a manifold might change if one weakens the definition 
of a manifold by dropping some of these constraints. As we will see in this little essay, these constraints do 
not ``come for free'' in the sense that they just follow from other impositions on the underlying spaces. Dropping these 
topological considerations has a major impact on analysis and hence physics that should be modelled with such spaces. 
For example, if one omits the constraint, that the underlying space has to be Hausdorff, one has to expect that converging 
sequences can admit multiple limit points:\newpage
\begin{theorem}[Hausdorffness and uniqueness of limits, \cite{wiki}]
    Let $X$ be a first-countable\footnote{A topological space is first-countable if every point has a countable local basis. 
    For example, every Hausdorff space which is locally Euclidean and connected is first countable} topological space. 
    Then the following are equivalent:
    \begin{itemize}
        \item $X$ is Hausdorff,
        \item every convergent sequence in $X$ has a unique limit.
    \end{itemize}
\end{theorem}
As this is a classical result in topology,\footnote{see \cite{wiki}} we will omit the proof. 
Converging sequences having multiple limit points seems like a rather inappropriate choice of topological spaces 
- especially when it comes to modelling (classical) physical systems with such spaces.
Furthermore, the following theorem holds
\begin{theorem}[Paracompactness and partition of unity, \cite{toenniessen}]
    A topological Hausdorff space is paracompact if and only if every open cover admits a subordinate partition of unity.
\end{theorem}
However, the partition of unity is an important tool for drawing certain local definitions to global ones. 
A classic example where the partition of unity is used is the proof that every manifold has a Riemannian metric. 
Moreover, the notion of integral on a manifold is often only made possible by the partition of unity. 
Therefore, if one cannot guarantee the existence of it, one would still have to justify why certain concepts such as 
the integral etc. are well-defined.

As outlined in the abstract of this essay, let us now study an example of a topological space that can be considered a 
``manifold'' if one drops the constraint of Hausdorffness and paracompactness:
\section{The line with many origins}
The line with many origins is a classical and famous topological space, as it serves as many counterexamples. 
We will now construct such a 'line with many origins' and we will see that it is neither Hausdorff nor paracompact, 
but it still fits the 'relaxed' definition of a manifold. Let us start with the definition:
\begin{definition}[The line with many origins]
    Let $I$ be some index set. We define the line of many origins as the topological space
    \begin{align*}
        L := \left(\coprod_{i\in I}\R\right)/_\sim
    \end{align*}
    where
    \begin{align*}
        (x, i)\sim (y,j) :\Leftrightarrow (x=y\land x\neq 0)
    \end{align*}
    for all $i,j\in I$ and $\R$ carries the usual topology induced by the metric $d(x,y) = \vert x-y\vert$.
\end{definition}
For shorthand notation, we denote $0_i$ for the origin in the $i$-th copy of $\R$. For points $x\in L$ that 
differ from the origins, the neighbourhood basis just coincides with the neighbourhood basis for $\R$ with the 
usual topology. Hence
\begin{align*}
    \{(-\varepsilon + x, x + \varepsilon):\varepsilon>0\}
\end{align*}
is the neighbourhood basis for such points. On the other hand, each $0_i\in L$ has its own ``copy'' of neighbourhoods 
from $\R$ around the origin. Hence
\begin{align*}
    \{(-\varepsilon,0)\cup\{0_i\}\cup(0,\varepsilon):\varepsilon>0\}
\end{align*}
is the neighbourhood basis for the origin $0_i$. Let $B_i(\varepsilon) := (-\varepsilon,0)\cup\{0_i\}\cup(0,\varepsilon)$ 
be the $\varepsilon$-ball around the origin $0_i$.
\begin{figure}[h]
    \centering
    \begin{tikzpicture}[scale=3]
        \draw (-2,0) -- (-1/8, 0) node[draw, inner sep=0.6pt, circle,at end, anchor=west]{};
        \draw (1/8, 0) -- (2, 0) node[draw, inner sep=0.6pt, circle,at start, anchor=east]{};
        \foreach \i in {1, ..., 4}
        {
        \fill (0, \i/4-5/8) node[fill, inner sep=0.6pt, circle]{};
        }
        \draw (0, -1/2) node{\vdots};
        \draw (0, 1/2) node[above]{\vdots};
    \end{tikzpicture}
    \caption{Illustration of the line with many origins}
\end{figure}
\begin{claim}
    The topological space $L$ satisfies the 'relaxed' definition of a manifold. (i.e. $L$ is locally Euclidean and 
    has a smooth differentiable structure)
\end{claim}
\begin{proof}
    We prove this claim in several steps:
    \begin{itemize}
        \item To show that $L$ is locally Euclidean, we only need to consider neighbourhoods around the origins. 
		Let $i\in I$ and $\varepsilon > 0$. There is an obvious bijection
        \begin{align*}
            \phi\colon L\supset B_i(\varepsilon)\to (-\varepsilon,\varepsilon)\subset\R.
        \end{align*}
        Wlog\footnote{the cases when $0\notin U$ are essentially boring as $\phi$ just becomes the identity on such elements.} 
	take some $\eta, \kappa < \varepsilon$ and set $U = (-\kappa, \eta)$. 
	Then $\phi^{-1}(U) = (-\kappa, 0)\cup\{0_i\}\cup(0,\eta)$ is open in $L$. 
	Hence we immediately see that $L$ is locally Euclidean.
        \item Let $i\in I$ and $L_i := L\setminus\{0_j:i\neq j\}$. Then we may define the continuous maps
        \begin{align*}
            \phi_i\colon L_i\to\R,\ x\mapsto \begin{cases}
                0,\qquad&\text{for } x\neq 0_i,\\
                x,\qquad&\text{otherwise}.
            \end{cases}
        \end{align*}
        Then $\{L_i, \phi_i\}_{i\in I}$ is an atlas for $L$. Furthermore,
        \begin{align*}
            \phi_i\circ\phi_j^{-1}\colon \R\setminus\{0\}\to \R\setminus\{0\}
        \end{align*}
        is differentiable of class $C^{\infty}$ for every $i,j\in I$ as $\phi_i$ and $\phi_j^{-1}$ just 
	become the identity functions on elements that are not $0$.
    \end{itemize}
\end{proof}
\newpage
\begin{claim}
    The topological space $L$ is not Hausdorff and, if the index set $I$ is infinite, then $L$ is not paracompact.
\end{claim}
\begin{proof}
    We again prove this claim in several steps:
    \begin{itemize}
        \item First, we show that $L$ is not Hausdorff. Let $i, j\in I$ with $i\neq j$. Let $U_i$ be some neighbourhood 
		of $0_i$ and $U_j$ be some neighbourhood of $0_j$ such that $U_i\cap U_j = \emptyset$. 
		We have seen that the neighbourhood basis of the origin $0_i$ is provided by $B_i(\varepsilon)$ 
		for some $\varepsilon>0$. Hence, there are $\varepsilon_i, \varepsilon_j> 0$ such 
		that $B_i(\varepsilon_i)\subset U_i$ and $B_j(\varepsilon_j)\subset U_j$. 
		But $B_i(\varepsilon_i)\cap B_j(\varepsilon_j)\neq\emptyset$ and hence $U_i\cap U_j\neq\emptyset$. 
		This is a contradiction and thus $L$ cannot be Hausdorff.
        \item Consider the open cover $(U_i)_{i\in I}$ of $L$ where
        \begin{align*}
            U_i := (-\infty,0)\cup\{0_i\}\cup(0,\infty).
        \end{align*}
        Let $(V_k)_{k}$ be some refinement of $(U_i)_i$. 
	Define $\mathcal{S}:= \{k\mid\exists i\in I: 0_i\in V_k\}$. Let $j\in I$ be fixed 
	and $U$ be some open neighbourhood of $0_j$. Then there is some $k'\in\mathcal{S}$ such 
	that $U\cap V_{k'}\neq\emptyset$. So there is $\varepsilon>0$ with $B_j(\varepsilon)\subset U\cap V_{k'}$. 
	But for every $k\in\mathcal{S}$ there is some $\delta>0$ such that $[(-\delta,\delta)\setminus\{0\}]\subset V_k$. 
	Hence 
        \begin{align*}
            \emptyset\neq B_j(\varepsilon)\cap V_k\subset U\cap V_k
        \end{align*}
        for all $k\in\mathcal{S}$. If $I$ is infinite, $\mathcal{S}$ must also be infinite. 
	Thus we conclude that $L$ can not be paracompact if $I$ is infinite.
    \end{itemize}
\end{proof}
\section{Conclusion}
We have seen that dropping topological constraints like Hausdorffness and paracompactness leads to pathological spaces with 
undesirable properties, as illustrated by the line with many origins. These constraints are therefore essential for ensuring 
well-behaved manifolds suitable for geometry, analysis, and physics.
\printbibliography
\end{document}
